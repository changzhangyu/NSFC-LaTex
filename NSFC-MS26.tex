%!TEX program = lualatex

\documentclass[12pt]{article}
\usepackage[UTF8]{ctex}
%\usepackage[fontset=ubuntu]{ctex}

\usepackage{nsfc}
\usepackage[misc]{ifsym}
\usepackage{pifont}
\usepackage{overpic}
\usepackage{subfigure}

\usepackage{color}

\usepackage{multirow}

\usepackage{pifont}
\newcommand{\cmark}{\ding{51}}%
\newcommand{\xmark}{\text{\ding{55}}}

\newcommand{\cmm}[1]{\textcolor[rgb]{0,0.6,0}{CMM: #1}}
\newcommand{\todo}[1]{{\textcolor{red}{\bf [#1]}}}
\newcommand{\mypara}[1]{\paragraph{#1.}}
\newcommand{\myEmph}[1]{\textbf{\textcolor[rgb]{0,0,0.25}{#1}}}
\newcommand{\addImg}[2]{\includegraphics[width=#1\textwidth]{#2}}
\newcommand{\addFloat}[3]{\subfigure[#1] {\addImg{#2}{#3}}}

\newcommand{\figref}[1]{图\ref{#1}}
\newcommand{\tabref}[1]{表\ref{#1}}
\newcommand{\equref}[1]{式\ref{#1}}
\newcommand{\secref}[1]{第\ref{#1}节}
\newcommand{\citess}[1]{\textsuperscript{\cite{#1}}}

\newcommand{\myCite}[1]{\textbf{#1}$^\text{我们}_\text{工作}$}

\WarningFilter{latexfont}{Font}
\WarningFilter{font}{Font}
\WarningFilter{fontspec}{Font}


\graphicspath{{figures/}{Imgs/}{Imgs/awards/}}


\begin{document}

%%%%%%%%% TITLE

\title{\emph{\zihao{2} 报告正文}}
\maketitle
\thispagestyle{empty}



\ContentDes{(一)立项依据:}

\NsfcNote{(为什么要开展此项研究,研究的科学技术价值如何)}

研究意义、国内外研究现状及发展动态分析,需结合科学研究发展趋势来论述科学意义;
或结合国民经济和社会发展中迫切需要解决的关键科技问题来论述其应用前景。

题目是你对评审专家说的第一句话。需要创新、创新、再创新!!
尽量回答“干什么、对象是什么、用什么方法、解决什么问题”。
简洁明确,具体清楚。不宜过长,不宜出现过多的关键词,但最好要有新意的关键词出现,
包含研究视角、方法和研究对象的创新,
最好能让专家一看到“题目名称”就能基本了解本申请重点要研究的问题。 
忌讳项目名称重复,即使所提出的与以前资助项目研究内容有所不同,甚至有所创新,
但名称重复很难给人以新意。
\myEmph{建议检索类似课题历年资助情况,避免重复!!}
%
在选择研究题目(或方向)时,应本着“扬长避短”的原则,尽量结合自己的研究基础;
缺乏一定科学(研究)基础的“创新”是不成立的,许多情况甚至是“空想”。
选题最好以问题为导向,不要新以技术、新方法的应用为导向!
忌盲目追求“学科前沿”和“研究热点”问题。


写本子最难的是\myEmph{想题目}和\myEmph{写摘要},
需要广泛快速阅读近几年的文献。
可以考虑找到一个具体的场景,用场景将本子的研究内容串起来。
也可以围绕一个关键科学问题,将多个方面的研究内容有机结合起来。
建议先画两个图:\myEmph{研究内容框架图}和\myEmph{详细技术路线图}。
有了这两个总蓝图,整个本子写起来会更加的有条理,避免各种混乱和不一致。
%
为了方便专家快速抓住重点,建议把\myEmph{重点内容标粗},
让专家即使只阅读非常少的加粗字体,也可以获得判断本子优劣的足够关键信息。


我在参加基金委重点项目会评之前,
基金委领导特意给评委们科普了基金委最看重的\myEmph{灵魂六问},请专家们重点关注:
\begin{itemize}
    \item 该项目想做什么?请用大同行能够理解的术语表达您的研究目标。
    \item 目前的做法有哪些局限性?
    \item 您的方法有什么独特性,为什么您认为它会成功?
    \item 谁关心该项目取得的成果?
    \item 如果您成功了,对该领域有的推动作用是什么?
    \item 中期和期末,如何检查该项目计划成功与否?
\end{itemize}

在撰写基金的过程中,各部分需要重点阐明
\begin{itemize}
    \item 立项依据:为什么做
    \item 研究内容:做什么
    \item 研究目标:做到什么程度
    \item 研究方案:如何做
    \item 工作基础等:我能做该项目
\end{itemize}


具体到立项依据,重点阐述清楚:
\myEmph{为什么要做这个课题?重大需求、存在问题、有解决思路。谁会关心该项目取得的成果?}
让评审者读了申请书以后要有如下感觉:
这个研究很重要,国内外都在做,但有要害问题没有解决,
申请人提出了很好的解决途径,思路很独特且合理,
若沿着这条思路做几个方面的研究,有解决希望。


对基础研究,着重结合国际科学发展趋势,论述项目科学意义对应用基础研究,
着重结合学科前沿,围绕国民经济和社会发展中的重要科技问题论述其应用前景。
%
立项依据部分应该包括:
\begin{itemize}
    \item 立项意义;
    \item 国外同类研究状况;
    \item 国内同类研究状况;
    \item 本课题组的研究基础和选题的依据。    
\end{itemize}
立项依据论述要简明扼要,有理有据。
要用准确的学术语言,将问题论述清楚,一般要考虑如下问题:
1、什么人在研究?研究了些什么?核心科学问题是什么? 
2、人家怎么进行研究?解决了些什么问题?还有什么问题没解决? 
哪些问题是别人想到了的?但没有解决?
3、你考虑怎么解决?哪些问题是别人还没有想到的?你又是考虑怎么来解决?
4、如果您成功了,对该领域有什么推动作用?



\section{研究背景及科学意义}

\subsection{*题目*对**具有重要作用}

第一段:主要描述题目的主要工作,解释题目中涉及的术语,
然后说明输入和输出,说明目前题目的主要工作是如何实现的,
具有什么样的意义。

第二段:说明第一段描述的内容的局限性和不足,
引出本项目的主要工作。

第三段:说明解决上述局限性和不足,有什么实际应用效果。

第四段:说明该项目工作对***的重要作用。


\subsection{*题目*具有挑战性的问题}

项目需要那些技术,这些技术的复杂程度,面临的主要问题,
要使用一些术语来表述,最后要有结论性语言,例如:

\myEmph{因此,当前***技术很难应用于面向***的研究,
在***等方面也缺少针对 *** 的专门研究,
特别是缺乏针对***的有效方法。
这些局限性限制了***分析技术的发展 。
}


\subsection{***思想为***题目*研究提供了新途径}

一些最新的技术,有利于项目工作的解决。
最后务必有一段总结的语言,例如

\myEmph{本项目基于***数据,建立***,利用***表示和分析方法,
以***为核心,研究***演化规律,
在理论和技术层面都具有重要的引领和示范作用。}


\section{***题目*面临的挑战}

提出3-4个挑战,分别对应后面的主要研究内容,最后一定要有对挑战的总结描述,例如

\myEmph{因此,面向****的***,需要对***、***和***进行统一的概念描述与框架建模。
在此基础上,通过****是一个可行途径。}


\section{总结}

说明课题组具有解决上述挑战的能力,例如:

\myEmph{综上所述,***,本项目从***出发,面对***等挑战,
研究***关键科学问题。
项目拟在以下三方面开展研究:****}


项目组在***具有扎实的研究基础,承担了多项国家和省部级课题, 典型的有:***。
这些项目的完成,使项目组在***方面积累了丰富的研究经验,
也为该项目的实施奠定了良好的技术基础。

本项目直接面向***国家战略,可以为***等重要应用提供核心技术支撑,
对推进***的应用和发展具有重要的意义。

\section{国内外相关工作}

如果申请人从未在所申请项目的研究领域发表过一篇论文,
或者申请书中对国内外研究现状阐述不明,不附主要参考文献目录,
说明申请人在这一研究领域无研究工作基础,不具备实施该项目的研究能力。

在相关工作的评述中,应该尽量广泛的包含各方面的先进成果:
既有国际上顶级的研究,也得有国内最先进的成果。
\myEmph{建议同时包含自己的中文\cite{21SC_WebSegE}和英文\cite{ChengZMHH10}的代表作,
这些相关的代表作也标明自己在该领域的研究基础。}
可以引用几个代表性图,形象的展示一些重要的背景知识,方便大同行理解。

可以对相关工作进行分组介绍。
每组介绍之后简单总结一下现有工作和拟研究工作的关系。
\myEmph{现有工作存在那些不足需要进一步研究。
这些总结性的结论建议粗体强调,方便评审人迅速理解。}

{
\bibliographystyle{nsfc.bst}
\bibliography{Cmm}
}





%%%%%%%%%%%%%%%%%%%%%%%%%%%%%%%%%%%%%%%%%%%%%%%%%
\clearpage
\ContentDes{(二)研究内容:}

\NsfcNote{(提纲不做限制,请按照研究工作的自身逻辑撰写。应提炼出特色与创新点、年度研究计划)}


\mypara{研究目标}

重点阐述本项目计划\myEmph{做到什么程度}。
一定要用大同行能够理解的术语来描述。


\section{总体思路}


重点阐述本项目计划\myEmph{做什么}。
如\figref{fig:teaser}所示,建议撰写之前仔细画一个主要研究内容的图。
我通常习惯用PowerPoint作图,做好后导出为pdf格式,既可以保持图片不会文件太大,也可以保证放大后非常清晰。
幻灯片直接导出的pdf可能存在空白边,可以用WPS中的“页面-剪裁页面”去掉白边。
作图用的pptx文件我也通常会保存起来,例如这个模版\LaTeX 文件中的“prepare/NSFC-Figs.pptx”。
如果是大的国基金项目,后续可能涉及答辩。
答辩时这个图可以用,保留pptx格式也方便到时候做尺寸和布局的调整。


\begin{figure}[ht]
	\centering
    \begin{overpic}[width=0.8\columnwidth]{framework.pdf}
    \end{overpic}
    \caption{本项目主要研究内容。
    }\label{fig:teaser}
\end{figure}




\section{拟解决的关键科学问题}



\section{拟采取的研究方案及可行性分析}{
(包括研究方法、技术路线、实验手段、关键技术等说明);}


\subsection{拟采取的技术路线}

如图\figref{fig:pipline}所示,建议在具体写技术路线之前,先厘清这个框架图。
技术路线的框架图通常比研究内容更加饱满,可以更好的展示本项目的研究思路。

\begin{figure}[ht]
	\centering
    \begin{overpic}[width=\columnwidth]{pipeline.pdf}
    \end{overpic}
    \caption{本项目的技术路线。
    }\label{fig:pipline}
\end{figure}


\subsection{可行性分析}

既然国内外相关工作都没能解决你提出的重要问题,为什么你觉得自己有望解决该问题。
论述的时候通常包括:独特的时机、与众不同的方案、雄厚的相关科研积累等。


\section{本项目的特色与创新之处}

您的方法有什么独特性,为什么您认为它会成功?

\section{年度研究计划及预期研究结果}

(包括拟组织的重要学术交流活动、国际合作与交流计划等)。

\subsection{年度研究计划}


\subsection{预期研究成果}

项目各阶段,特别是中期和期末,如何检查该项目计划成功与否?





%%%%%%%%%%%%%%%%%%%%%%%%%%%%%%%%%%%%%%%%%%%%%%%%%
\ContentDes{(三)研究基础:}


\NsfcNote{\textbf{$1$. 研究基础与可行性分析}(与本项目相关的研究工作积累和已取得的研究工作成绩,研究风险的应对措施等);}

国家正在大力推动国产人工智能生态的发展。如果有基于国产生态的开源基础,建议重点突出。

\NsfcNote{\textbf{$2$. 工作条件}(包括已具备的实验条件,尚缺少的实验条件和拟解决的途径,包括利用国家实验室、全国重点实验室和部门重点实验室等研究基地的计划与落实情况);}

无

\NsfcNote{\textbf{$3$. 正在承担的与本项目相关的科研项目情况}(申请人和主要参与者正在承担的与本项目相关的科研项目情况,包括国家自然科学基金的项目和国家其他科技计划项目,要注明项目的资助机构、项目类别、批准号、项目名称、获资助金额、起止年月、与本项目的关系及负责的内容等);}

无

\NsfcNote{\textbf{$4$. 完成国家自然科学基金项目情况}(对申请人负责的前一个已资助期满的科学基金项目(项目名称及批准号)完成情况、后续研究进展及与本申请项目的关系加以详细说明。另附该项目的研究工作总结摘要(限500字)和相关成果详细目录)。}


无
  


%%%%%%%%%%%%%%%%%%%%%%%%%%%%%%%%%%%%%%%%%%%%%%%%%
\ContentDes{(四)其他需要说明的情况:}

\NsfcNote{$1$. 申请人同年申请不同类型的国家自然科学基金项目情况(
列明同年申请的其他项目的项目类型、项目名称信息,
并说明与本项目之间的区别与联系)。}

无

\NsfcNote{$2$. 具有高级专业技术职务(职称)的申请人是否存在同年申请
或者参与申请国家自然科学基金项目的单位不一致的情况;
如存在上述情况,列明所涉及人员的姓名,申请或参与申请的其他项目的项目类型、
项目名称、单位名称、上述人员在该项目中是申请人还是参与者,并说明单位不一致原因。}

无

\NsfcNote{$3$. 具有高级专业技术职务(职称)的申请人是否存在与
正在承担的国家自然科学基金项目的单位不一致的情况;
如存在上述情况,列明所涉及人员的姓名,正在承担项目的批准号、项目类型、
项目名称、单位名称、起止年月,并说明单位不一致原因。}{}

无

\NsfcNote{$4$. 其他。}{}

无

\end{document}
